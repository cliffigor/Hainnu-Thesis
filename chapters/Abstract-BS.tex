%%==================================%%
%%          生成本科生标题           %%
%%			    无需修改			%%
%%==================================%%
\makecntitle
%%==================================%%
%%             中文摘要             %%
%%==================================%%
\abstract
% \lipsum[xx] 和 \zhlipsum[xx][xx] 等命令为测试使用的随机文本命令
% 使用模板写作时请删除,下同
{\zhlipsum[1-3][name=aspirin]
}

%% ==== 创新点
%\begin{inovation}
% \item \zhlipsum[10][name=zhufu]
% \item \zhlipsum[11][name=zhufu]
% \item \zhlipsum[12][name=zhufu]
%\end{inovation}

%% ==== 关键词
\keywords{课程知识;建构;建构主义;社会建构;个体建构}

%%==================================%%
%%          生成本科生标题           %%
%%			    无需修改			%%
%%==================================%%
\makeentitle

%%==================================%%
%%             英文摘要             %%
%%==================================%%
%% 使用带选项[Abstract]\abstract命令添加英文摘要
\abstract[Abstract]
% ==== 随机文本,使用模板时请删除
{\lipsum[2-3]}
%% ==== 英文创新点
%\begin{inovation}[Innovation]
% \item \lipsum[3][1]
% \item \lipsum[3][2-3] 
% \item \lipsum[3][4-6] 
%\end{inovation}
% ==== 英文关键词
\keywords[Key words]{Curricula Knowledge; Construction; Constructivism; Social Constructivism; Individual Constructivism}