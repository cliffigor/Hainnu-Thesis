\usepackage{listings}

\lstset{
    basicstyle=\ttfamily,   %使用默认的等宽,而非texlive调用的中文字体
    escapeinside=``,        %使用 “逃逸” 字串来显示中文
    escapeinside={(*@}{@*)},%在代码中使用LaTeX
    tabsize=4, 
	numbers=left,
	numberstyle=\tiny,
	commentstyle=\ttfamily\color{red!80},
%	frame=shadowbox,
	rulesepcolor=\color{red!20!green!20!blue!20},          
	flexiblecolumns=true, %
	breaklines=true, %对过长的代码自动换行
	breakautoindent=true, %
	breakindent=4em, %
    keywordstyle=\color{blue!90}\bfseries,         %代码关键字的颜色为蓝色, 粗体
%    backgroundcolor=\color[RGB]{245,245,244},            % 设定背景颜色
%    keywordstyle=\color[RGB]{40,40,255},                 % 设定关键字颜色
%    numberstyle=\footnotesize\color{darkgray},           % 设定行号格式
%    commentstyle=\it\color[RGB]{0,96,96},                % 设置代码注释的格式
    stringstyle=\slshape\color[RGB]{128,0,0},   % 设置字符串格式
    showstringspaces=false,                              % 不显示字符串中的空格
%	language=c++,                                        % 设置语言
}



%% ++++++++++++++++++++++++++++++++++++++++++++++++
% Algorithm
%% ++++++++++++++++++++++++++++++++++++++++++++++++

%伪代码
\usepackage{algorithm}         
\usepackage{algorithmicx}
\usepackage{algpseudocode}
\usepackage{float}
\counterwithin{algorithm}{chapter}
\renewcommand{\algorithmicrequire}{\textbf{输入:}}
\renewcommand{\algorithmicensure}{\textbf{输出:}}

% 添加do-while语句
\renewcommand{\algorithmicrepeat}{\textbf{do}}
\renewcommand{\algorithmicuntil}{\textbf{while}}

% 添加switch-case 语句
\algnewcommand\algorithmicswitch{\textbf{switch}}
\algnewcommand\algorithmiccase{\textbf{case}}
\algnewcommand\algorithmicdefault{\textbf{default}}
\algnewcommand\algorithmicassert{\texttt{assert}}
\algnewcommand\Assert[1]{\State \algorithmicassert(#1)}%
% New "environments"
\algdef{SE}[SWITCH]{Switch}{EndSwitch}[1]{\algorithmicswitch\ #1\ \algorithmicdo}{\algorithmicend\ \algorithmicswitch}%
\algdef{SE}[CASE]{Case}{EndCase}[1]{\algorithmiccase\ #1}{\algorithmicend\ \algorithmiccase}%
\algdef{SE}[DEFAULT]{Default}{EndDefault}[1]{\algorithmicdefault\ #1}{\algorithmicend\ \algorithmicdefault}%

%%%%%%%%%%%%%%%%%%%%%%%%%%%%%%%%%%%%%%%%%%%%%%%%%%%%%%%%%%%%%%%
% 伪代码分页
%%%%%%%%%%%%%%%%%%%%%%%%%%%%%%%%%%%%%%%%%%%%%%%%%%%%%%%%%%%%%%%
\makeatletter
\renewcommand{\ALG@name}{算法}
\newenvironment{breakablealgorithm}
{% \begin{breakablealgorithm}
	\begin{center}
		\refstepcounter{algorithm}% New algorithm
		%\wuhao 五号字大小
		\setlength{\baselineskip}{15pt}  %定义行间距
		% \@fs@pre for \@fs@ruled 画线
		\renewcommand{\caption}[2][\relax]{% Make a new \caption
			\hrule height.8pt depth0pt \kern0pt
			{\raggedright\textbf{\ALG@name~\thealgorithm} ##2\par}%
			\ifx\relax##1\relax % #1 is \relax
			\addcontentsline{loa}{algorithm}{\protect\numberline{\thealgorithm}##2}%
			\else % #1 is not \relax
			\addcontentsline{loa}{algorithm}{\protect\numberline{\thealgorithm}##1}%
			\fi
			\kern2pt\hrule\kern2pt
		}
	}{% \end{breakablealgorithm}
		\kern2pt\hrule\relax% \@fs@post for \@fs@ruled 画线
	\end{center}
}


\newcommand{\cpvar}[1]{\texttt{#1}}
\newcommand{\cpfile}[1]{\texttt{#1}}
\newcommand{\cpfun}[2]{\ProcName{#2}}
\newcommand{\True}{\texttt{TRUE}}
\newcommand{\False}{\texttt{FALSE}}
\newcommand{\PtrToFun}[1]{\texttt{#1}}
\newcommand{\ProcName}[1]{\textsc{#1}}